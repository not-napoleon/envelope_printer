% \iffalse meta-comment
%
% Copyright 1994-2004 Johannes Braams.  All rights reserved.
% 
% This file is part of the mailing package.
% -----------------------------------------
% 
% It may be distributed and/or modified under the
% conditions of the LaTeX Project Public License, either version 1.3
% of this license or (at your option) any later version.
% The latest version of this license is in
%   http://www.latex-project.org/lppl.txt
% and version 1.3 or later is part of all distributions of LaTeX
% version 2003/12/01 or later.
% 
% This work has the LPPL maintenance status "maintained".
% 
% The Current Maintainer of this work is Johannes Braams.
% 
% The list of all files belonging to the mailing package is
% given in the file `manifest.txt'.
% 
% The list of derived (unpacked) files belonging to the distribution
% and covered by LPPL is defined by the unpacking scripts (with
% extension .ins) which are part of the distribution.
% \fi
% \CheckSum{157}
%
%\iffalse
%% Copyright (C) 1999-2004 Johannes L. Braams texniek at braams.cistron.nl
%%
%%
%<*dtx>
\ProvidesFile{mailing.dtx}
%</dtx>
\NeedsTeXFormat{LaTeX2e}
%<package>\ProvidesPackage{mailing}
%<driver>\ProvidesFile{mailing.drv}
%\fi
%\ProvidesFile{mailing}
    [2004/02/20 v1.0b Send letters to multiple addresses]
%\iffalse
%<*driver>
\documentclass{ltxdoc}
\begin{document}
\pagestyle{myheadings}
\DocInput{mailing.dtx}
\end{document}
%</driver>
%\fi
%
% \GetFileInfo{mailing}
% \title{Creating a mailing}
% \author{Johannes Braams}
% \date{september 7, 1994}
% \maketitle
%
% \markboth
%      {mailing package version \fileversion\space as of \filedate}
%      {mailing package version \fileversion\space as of \filedate}
%
% \section{Introduction}
%
%    This package is intended to be used when you want to send a large
%    number of letters, all with (almost) the same text.
%
%    This package is based on the former style option \texttt{merge}
%    by Graeme McKinstry, but is is a reimplementation with a
%    different user interface.
%
% \section{The user interface}
%
%  \DescribeMacro{\addressfile}
%    The commands |\addressfile| takes a filename as its argument.
%    When the file can't be found by \LaTeX, the user will be asked to
%    supply a different name.
%
%    The address file should have the following format:
% \begin{verbatim}
%Name of addressee
%Street\\town
%Opening text of the letter
%(optional definitions)
%<blank line>
%Name of addressee
%Street\\town
%Opening text of the letter
%(optional definitions)
%<blank line>
%...
% \end{verbatim}
%    The various addresses are separated by a blank line in the file
%    (multiple blank lines in between addresses are allowed).
%    It is also possible to have multiple lines with definitions; they
%    will all be executed.
%
%  \DescribeMacro{\mailingtext}
%    The text of the mailing is entered as the argument of
%    |\mailingtext|. A difference with the original \texttt{merge.sty}
%    is that this package allows control sequences in the argument of
%    |\mailingtext|. These control sequences should then be defined in
%    the file with the address information.
%
%  \DescribeMacro{\makemailing}
%    When |\makemailing| is called the letters are produced, combining
%    the information found in the address file with the text of the
%    mailing. 
%
% \StopEventually{}
%
%  \section{The implementation}
%
%  \subsection{User interface}
%
%  \begin{macro}{\addressfile}
%    The argument to |\adressfile| is the name of the file with the
%    address information.
%    \begin{macrocode}
\newcommand{\addressfile}[1]{%
  \def\M@filename{#1}}
%    \end{macrocode}
%  \end{macro}
%
%  \begin{macro}{\mailingtext}
%    The argument to this macro contains the entire text of the
%    mailing. The text may contain control sequences to be able to
%    make variations in the text.
%    \begin{macrocode}
\long\def\mailingtext#1{\global\mailing@text={#1}}
%    \end{macrocode}
%  \end{macro}
%
%  \begin{macro}{\makemailing}
%    The command |\makemailing| will produce the mailing, reading
%    addresses, openings and optional definitions of variable text
%    parts from an external file.
%    \begin{macrocode}
\def\makemailing{%
  \M@openadrfile
  \loop
    \read@info
    \if@notready
      \begin{letter}{\M@toname\\\M@toaddress}%
        \opening{\M@opening}%
        \vskip\baselineskip
        \the\mailing@text
      \end{letter}
    \fi
    \if@notready
  \repeat}
%    \end{macrocode}
%  \end{macro}
%
%  \subsection{Allocations}
%
%  \begin{macro}{\M@adrfile}
%    We need to allocate an input stream for the file with the address
%    information. 
%    \begin{macrocode}
\newread\M@adrfile
%    \end{macrocode}
%  \end{macro}
%
%  \begin{macro}{\mailing@text}
%    The contents of the letter are stored in a token register
%    \begin{macrocode}
\newtoks\mailing@text
%    \end{macrocode}
%  \end{macro}
%
%  \begin{macro}{\if@notready}
%    A switch which indicates if the file |\M@adrfile| has been
%    exhausted. 
%    \begin{macrocode}
\newif\if@notready
\newif\if@notemptyoreof
%    \end{macrocode}
%  \end{macro}
%
%  \subsection{Internal macros}
%
%  \begin{macro}{\M@openadrfile}
%    The macro |\M@openadrfile| tries to open |\M@filename|. It that
%    doesn't succeed it asks the user to supply a new name. This is
%    done untill a file is found.
%    \begin{macrocode}
\def\M@openadrfile{%
  \openin\M@adrfile\M@filename\relax
  \ifeof\M@adrfile
    \loop
      %\PackageWarning{mailing}{I can't find the file \M@filename}
      \typeout{I can't find the file \M@filename!}
      \closein\M@adrfile
      \typein[\M@filename]{Enter a new name}
      \openin\M@adrfile\M@filename
      \ifeof\M@adrfile
    \repeat
  \fi}
%    \end{macrocode}
%  \end{macro}
%
%  \begin{macro}{\read@info}
%    The macro |\read@info| takes care of the reading of all the
%    information from |\M@adrfile|, needed to format another letter.
%    \begin{macrocode}
\def\read@info{%
  \skip@empty@lines
%    \end{macrocode}
%    The macro |\skip@empty@lines| leaves the non-empty line it found
%    in |\M@lines|. If it found an end of file condition the
%    |\if@notready| flag will be set to |\iffalse|.
%    \begin{macrocode}
  \if@notready
    \let\M@toname\M@line
    \read\M@adrfile to\M@toaddress
    \read\M@adrfile to\M@opening
    \test@eof
    \if@notready\read@defs\fi
  \fi
  }
%    \end{macrocode}
%  \end{macro}
%
%  \begin{macro}{\read@defs}
%    Reads definitions of control sequences from the file |\M@adrfile|
%    until either an empty line is found or the end of file is
%    reached. Each line is stored in a control sequence and it is
%    executed after all definitions are read.
%    \begin{macrocode}
\def\read@defs{%
  \def\M@defns{}%
  {\loop
    \endlinechar=-1
    \read\M@adrfile to\M@line
    \endlinechar=`\^^M
%    \end{macrocode}
%    We need to get the expansion of |\M@line| into the definition of
%    |\M@defns|, not |\M@line| itself. Therefore |\M@line| is expanded
%    before |\M@defns|.
%    \begin{macrocode}
    \expandafter\toks@\expandafter\expandafter
      \expandafter{\expandafter\M@defns\M@line}%
    \xdef\M@defns{\the\toks@}%
    \test@emptyoreof
    \if@notemptyoreof
    \repeat}%
  \M@defns
  }
%    \end{macrocode}
%  \end{macro}
%
%  \begin{macro}{\test@eof}
%    The macro |\test@eof| tests the status of of the input file. 
%    \begin{macrocode}
\def\test@eof{%
  \ifeof\M@adrfile
    \@notreadyfalse
  \else
    \@notreadytrue
  \fi}
%    \end{macrocode}
%  \end{macro}
%
%  \begin{macro}{\test@emptyoreof}
%    The macro |\test@emptyoreof| checks whether we reached an empty
%    line \emph{or} the end of the file.
%    \begin{macrocode}
\def\test@emptyoreof{%
  \@notemptyoreoftrue
  \ifx\M@line\@empty
    \global\@notemptyoreoffalse
  \fi
  \ifeof\M@adrfile
    \global\@notemptyoreoffalse
    \global\@notreadyfalse
  \fi}
%    \end{macrocode}
%  \end{macro}
%
%  \begin{macro}{\skip@empty@lines}
%    This macro skips empty lines until it finds either a non-empty
%    line or the end of the file. If necessary it sets the
%    |\if@notready| flag. The last line read is left in |\M@line|.
%    \begin{macrocode}
\def\skip@empty@lines{%
  {\loop
    \endlinechar=-1
    \ifeof\M@adrfile
      \global\@notreadyfalse
      \@tempswafalse
    \else
      \global\@notreadytrue
      \global\read\M@adrfile to\M@line
      \ifx\M@line\@empty
        \@tempswatrue
      \else
        \@tempswafalse
      \fi
    \fi
    \if@tempswa
    \repeat}%
  }
%    \end{macrocode}
%  \end{macro}
%
% \Finale
\endinput